%		FONT AND INPUT ENCODING 
%
\usepackage[T1]{fontenc}
\usepackage[utf8]{inputenc}


%		LANGUAGE SETTINGS
%
\usepackage[ngerman]{babel} 	%German Language 
\usepackage[german=quotes]{csquotes} 	%Enable \enquote{} Statement for language-specific quotes 

%
%

%Für englische Arbeiten 
%\usepackage[ngerman]{english}

%		BIBLIOGRAPHY SETTINGS
%
\usepackage[backend=biber, autocite=footnote, style=authoryear-icomp, dashed=false]{biblatex} 	%Define Bibliography style 
\DefineBibliographyStrings{ngerman}{  %Change u.a. to et al. (german only!)
	andothers = {{et\,al\adddot}},             
} 
\setlength{\bibparsep}{\parskip}		%add some space between biblatex entries in the bibliography
\addbibresource{bibliography.bib}	%Add file bibliography.bib as biblatex resource 


%		GLOSSARIES AND ACRONYMS
%
%\usepackage[acronym, nomain, nopostdot, nonumberlist, toc, xindy={language=german}, shortcuts]{glossaries}
\usepackage[acronym, nopostdot, nonumberlist, toc, shortcuts]{glossaries}
\renewcommand{\glsgroupskip}{}

 \newglossarystyle{abkverz}{\setglossarystyle{super} 	
 	\renewcommand{\glossentry}[2]{% 
 		\hspace{-2mm}\textsf{\textbf{\glstarget{##1}{\glossentryname{##1}}}}% 
 		& \glossentrydesc{##1}% 
 		\tabularnewline % end of row
 	}%
 }

\setlength{\glsdescwidth}{.85\textwidth}
\setglossarystyle{abkverz}

\makeglossaries


%		LISTINGS
\usepackage{listings}	%Format Listings properly 
\lstset{numbers=left,
	numberstyle=\tiny,
	captionpos=b,
	basicstyle=\ttfamily\small}


%		EXTRA PACKAGES 
\usepackage{lipsum}    %Blind Text Generation
\usepackage{graphicx} %Include graphics 
\usepackage[german]{varioref} 	%Cross References made easy with \vref
\usepackage{caption}	%Better Captions 
\usepackage{todonotes} 	%Enter TODOs directly in text
\usepackage{booktabs} %nicer tables 


%		FONT SELECTION: Either use latin modern or Times / Helvetica combination
\usepackage{lmodern} %Latin modern font 
%\usepackage{mathptmx}  %Helvetica / Times New Roman fonts (2 lines)
%\usepackage[scaled=.92]{helvet} %Helvetica / Times New Roman fonts (2 lines)


%		PAGE HEADER / FOOTER 
\RequirePackage[automark,headsepline,footsepline]{scrpage2}
\pagestyle{scrheadings}
\renewcommand*{\pnumfont}{\upshape\sffamily}
\renewcommand*{\headfont}{\upshape\sffamily}
\renewcommand*{\footfont}{\upshape\sffamily}
\renewcommand{\chaptermarkformat}{}

\clearscrheadfoot

\ifoot[DHBW Mannheim]{DHBW Mannheim}
\ofoot[\pagemark]{\pagemark}

\ihead{\chaptername~\thechapter}
\ohead{\headmark}





