% !TEX root =  master.tex
\chapter{Einleitung}

\nocite{*}

Dieses Kapitel enthält die Einleitung mit ihren verschiedenen Abschnitten/Sections und Unterabschnitten.

\section{Beispiel Abschnitt: \LaTeX-Installation}

Zur Verwendung von \LaTeX-Installation einer Distribution z.~B.~TeXLive, MikTex etc.~sowie eines Editors z.~B.~TeXStudio, TeXnicCenter etc.~notwendig.

Installieren Sie zun\"achst die Distribution und anschließend den Editor. Beim ersten Start des Editors \"offnet sich ein 
Konfigurationsassistent, der zun\"achst nach dem Pfad der installierten Distribution fragt. 

Nach der Installation können k\"onnen Einstellungen z.~B.~f\"ur einen PostScript-Viewer gemacht werden. 
Dieser Schritt kann ohne Weiteres \"ubersprungen werden. Entscheidend sind die Einstellungen f\"ur den pdf-Viewer. 

Jetzt kann \LaTeX~verwendet werden. Um die Ausgabe eines Dokumentes zu erzeugen, muss das Dokument kompiliert werden (Ausgabe >
Aktives Dokument > Erstellen und betrachten).

\subsection{Beispiel Unterabschnitt: Aufbau eines \LaTeX-Dokuments}

Ein \LaTeX-Dokument besteht in der Regel aus folgenden Komponenten:
\begin{itemize}
	\item Pr\"aambel
	\item Titelseite
	\item Textteil
\end{itemize}

\subsection{Beispiel Unterabschnitt auf zweiter Ebene: Pr\"aambel}
In der Pr\"aambel werden global die Einstellungen f\"ur das gesamte Dokument definiert. Hierbei k\"onnen z.~B.~die Seitenr\"ander, 
der Zeilenabstand oder auch die Sprache f\"ur die Silbentrennung festgelegt werden. In der ersten Zeile eines jeden Dokumentes wird dabei
immer die zu verwendende Klasse festgelegt. Standardm\"aßig kann hier die Artikel-Klasse gew\"ahlt werden:

\texttt{\textbackslash documentclass[12pt,titlepage]\{article\}}

In den eckigen Klammern wird dabei u.a. die Standardschriftgr\"o\ss e f\"ur das gesamte Dokument festgelegt. 

Au\ss erdem werden in der Pr\"aambel die f\"ur das Dokument ben\"otigten Pakete festgelegt. Gebr\"auchlich sind vor allem folgende Pakete:
{\texttt{
\begin{itemize}
	\item \textbackslash usepackage[ngerman]\{babel\}
	\item \textbackslash usepackage[latin1]\{inputenc\}
	\item \textbackslash usepackage\{color\}
	\item \textbackslash usepackage[a4paper]\{geometry\}
	\item \textbackslash usepackage\{amssymb\}
	\item \textbackslash usepackage\{amsthm\}
	\item \textbackslash usepackage\{graphicx\}
\end{itemize}
}

Im vorliegenden Fall werden die Pakete in der Konfigurationsdatei \texttt{config.tex} festgelegt, deren Inhalt durch 
\texttt{\textbackslash input\{config\}} in das Hauptdokument \texttt{master.tex} inkludiert wird.

\subsubsection{Beispiel Unterabschnitt auf zweiter Ebene: Titelseite}

Nachdem die Dokumenten-Klasse und die zu verwendenden Pakete festgelegt worden sind,
folgt die Titelseite. Da die Titelseite bereits Teil des eigentlichen Dokuments ist, muss ihr
unbedingt der Befehl \texttt{\textbackslash begin\{document\}} vorausgehen. Am Ende des Dokuments sollte der Befehl
\texttt{\textbackslash end\{document\}} gesetzt werden. Alles was nach diesem Befehl steht, wird vom Compiler nicht mehr beachtet.

\section{Noch ein Beispiel-Abschnitt}

Der Textteil beinhaltet nun den eigentlichen Text des Dokuments.
